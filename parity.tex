\documentclass[twoside,colorback,accentcolor=tud4c,11pt]{tudreport}
\usepackage{tablefootnote}
\usepackage{ngerman}
\usepackage[utf8]{inputenc} 
\usepackage[T1]{fontenc}
\usepackage{graphicx}
\usepackage{isotope}
\usepackage{tabularx}
\usepackage{tabulary}
\usepackage{float}
\usepackage{siunitx}
\usepackage{hyperref}
\usepackage{siunitx}
\usepackage{units}
\usepackage{upgreek}
\usepackage{biblatex}
\usepackage[figure]{hypcap}
\usepackage{multirow}
\usepackage{amsmath}
\usepackage{subfig}
\usepackage{xfrac}
\usepackage{gensymb}
\usepackage{braket}

\title{Paritätsverletzung beim Beta-Zerfall}
\subtitle{	\begin{tabular}{p{8cm}ll}
Benedikt Paul Schallmo   &   Jonas Fischer \\ Matrikelnummer: 2686286  &   Matrikelnummer: 2240758       \\ email: \textaccent{ benediktschallmo@yahoo.de} & email: \textaccent{jonas.fischer.42gmail.com}  
			\end{tabular} }
\subsubtitle{ \\Versuchsbetreuung: Kristian König\\ Datum der Durchführung: 24.04.2017 \\ Abgabetermin: 16.05.2017}
\institution{Praktikum für Fortgeschrittene}
\sponsor{Hiermit erklären wir, dass wir die vorliegende Arbeit bzw. Leistung eigenständig, ohne fremde Hilfe und nur unter Verwendung der angegebenen Hilfsmittel angefertigt haben. Alle übernommenen Textstellen aus der Literatur beziehungsweise dem Internet sind als solche kenntlich gemacht. Diese Arbeit hat in gleicher oder ähnlicher Form noch keiner Prüfungsbehörde vorgelegen. \\\\ 
\begin{tabular}{lp{2em}lp{2em}l}
 \hspace{4cm}   && \hspace{4cm}  && \hspace{4cm}
 \\\cline{1-1}\cline{3-3}\cline{5-5}
 Ort, Datum     && Benedikt Schallmo && Jonas Fischer
\end{tabular}  }


\dedication{}
\lowertitleback{}
\listfiles
    
\begin{document}

\maketitle 

\tableofcontents


\chapter{Einleitung und Ziel des Versuchs bene}

\chapter{Physikalische Grundlagen}
\section{Radioaktive Zerfälle bene}
\section{Die vier Wechselwirkungen bene}
prozesse vergleich erhaltungsgrößen
\section{Parität jonas}

\section{Bremsstrahlung jonas}
\section{Wechselwirkung von elektromagnetischer Strahlung mit Materie jonas}
\section{Szintillationsdetektoren jonas}
\section{Wu-Experiment bene}
Einen Nachweis der Paritätsverletzung bei der schwachen Wechselwirkung liefert das Wu-Experiment. Dabei wird der Betazerfall an \isotope[60]{Co} untersucht, welches zu \isotope[60]{Ni}

\section{Zerfallschema bene}
	
\chapter{Durchführung und Aufbau bene}
Der Aufbau ist in Abbildung    dargestellt. 
  
     	
\chapter{Auswertung}
\section{Natrium jonas}
\subsection{Peaks erklären und Kanäle zusammenfassen}
\subsection{Energiekalibration}
\subsection{Apperativ vorgetäuschte Polarisation}
\section{Strontium}
\subsection{Bremsstrahlungsspektrum}
\subsection{Enpunktsenergie im Bremsstrahlungsspektrum}
\subsection{Polarisationsgrad}
\section{Fehlerberechnung und Messzeitabschätzung}

\chapter{Fazit}


\chapter{Anhang}





		

\renewcommand{\bibname}{Literaturverzeichnis}
\begin{thebibliography}{Bak89}



\end{thebibliography} 	



\end{document} 