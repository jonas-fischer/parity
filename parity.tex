\documentclass[twoside,colorback,accentcolor=tud4c,11pt]{tudreport}
\usepackage{tablefootnote}
\usepackage{ngerman}
\usepackage[utf8]{inputenc} 
\usepackage[T1]{fontenc}
\usepackage{graphicx}
\usepackage{isotope}
\usepackage{tabularx}
\usepackage{tabulary}
\usepackage{float}
\usepackage{siunitx}
\usepackage{hyperref}
\usepackage{siunitx}
\usepackage{units}
\usepackage{upgreek}
\usepackage{biblatex}
\usepackage[figure]{hypcap}
\usepackage{multirow}
\usepackage{amsmath}
\usepackage{subfig}
\usepackage{xfrac}
\usepackage{gensymb}
\usepackage{braket}

\title{Paritätsverletzung beim Beta-Zerfall}
\subtitle{	\begin{tabular}{p{8cm}ll}
Benedikt Paul Schallmo   &   Jonas Fischer \\ Matrikelnummer: 2686286  &   Matrikelnummer: 2240758       \\ email: \textaccent{ benediktschallmo@yahoo.de} & email: \textaccent{jonas.fischer.42gmail.com}  
			\end{tabular} }
\subsubtitle{ \\Versuchsbetreuung: Kristian König\\ Datum der Durchführung: 24.04.2017 \\ Abgabetermin: 16.05.2017}
\institution{Praktikum für Fortgeschrittene}
\sponsor{Hiermit erklären wir, dass wir die vorliegende Arbeit bzw. Leistung eigenständig, ohne fremde Hilfe und nur unter Verwendung der angegebenen Hilfsmittel angefertigt haben. Alle übernommenen Textstellen aus der Literatur beziehungsweise dem Internet sind als solche kenntlich gemacht. Diese Arbeit hat in gleicher oder ähnlicher Form noch keiner Prüfungsbehörde vorgelegen. \\\\ 
\begin{tabular}{lp{2em}lp{2em}l}
 \hspace{4cm}   && \hspace{4cm}  && \hspace{4cm}
 \\\cline{1-1}\cline{3-3}\cline{5-5}
 Ort, Datum     && Benedikt Schallmo && Jonas Fischer
\end{tabular}  }


\dedication{}
\lowertitleback{}
\listfiles
    
\begin{document}

\maketitle 

\tableofcontents


\chapter{Einleitung und Ziel des Versuchs bene}

\chapter{Physikalische Grundlagen}
\section{Radioaktive Zerfälle bene}
\section{Die vier Wechselwirkungen bene}
prozesse vergleich erhaltungsgrößen
\section{Parität}
Parität beschreibt das Verhalten einer physikalischen Größe unter Raumspiegelung. So gilt etwa für den Ortsvektor $ \vec{r} $
\begin{align}
\hat{P}\ket{\vec{r}}=-\ket{\vec{r}}
\end{align}
Dies nennt man negative Parität. Wenn sich das Vorzeichen nicht ändert besitzt diese Größe positive Parität, wie etwa der Drehimpuls $ \vec{L} $
\begin{align}
\vec{L}=m\vec{r}\times\vec{v}
\end{align}
Da sich sowohl bei $ \vec{r} $ als auch bei $ \vec{v} $ das Vorzeichen unter Raumspiegelung ändert ist 
\begin{align}
\hat{P}\ket{\vec{L}}=+\ket{\vec{L}}
\end{align}
\subsection{Polare und axiale Vektoren, Skalare und Pseudoskalare}
Als polaren Vektor bezeichnet man Vektoren, die unter Raumspiegelung ihr Vorzeichen ändern.\\
Als axiale Vektoren bezeichnet man folglich solche die dies nicht tun.\\
Ähnlich verhält es sich mit den Skalaren und Pseudoskalaren. Skalare ändern ihr Vorzeichen nicht, Pseudoskalare hingegen schon (siehe Tabelle \ref{tab:polar}).
\begin{table}[H]
\centering
\begin{tabular}{|c|c|c|}
\hline 
 & Parität + & Parität - \\ 
\hline 
Vektor & \shortstack{axial\\$\vec{\omega},\vec{L}$} & \shortstack{polar \\$\vec{v},\vec{x}$} \\ 
\hline 
Skalar & \shortstack{Skalar \\$m,T$} & \shortstack{Pseudoskalar\\Helizität $h$} \\ 
\hline 
\end{tabular} 
\caption{Beispiele für Polare und axiale Vektoren, Skalare und Pseudoskalare. Die Helizität $h$ ist definiert als $h=\vec{J}\cdot\frac{\vec{p}}{|\vec{p}|}$}\label{tab:polar}
\end{table}
Für uns kommen also nur polare Vektoren oder Pseudoskalare in Frage, wenn wir eine Paritätsverletzung messen wollen. Es bietet sich für uns an Pseudoskalare zu messen, da dies einfacher ist als Vektoren zu bestimmen.
\subsection{$\Theta$-$\tau$-Rätsel}
Lange nahm man an, dass die Parität eine Erhaltungsgröße sei, wie etwa die Energie oder der Impuls. Die Entdeckung des $ \Theta^+ $- und des $ \tau^+ $-Teilchen stellen dies in Frage. Die gleichen Eigenschaften der Teilchen legen nahe, dass es sich um ein und das selbe Teilchen handelt, jedoch zerfallen diese unterschiedlich:
\begin{align}
\Theta^+&\rightarrow\pi^++\pi^0\\
\tau^+&\rightarrow\pi^++\pi^++\pi^-
\end{align}
was an sich noch kein Problem ist, allerdings besitzen die Pionen alle eine negative Parität, womit sich für $\Theta^+$ eine positive und für $\tau^+$ eine negative Parität ergibt. Damit müssen die Teilchen als unterschiedliche behandelt werden oder die Parität ist beim Zerfall des \glqq $\Theta$-$\tau$-Teilchens\grqq\, nicht erhalten.
\section{Bremsstrahlung}
Bremsstrahlung bezeichnet die elektromagnetische Strahlung, die entsteht, wenn elektrische Ladung beschleunigt wird. In unserem Fall handelt es sich um ein Abbremsen in Materie. Die Elektronen werden im Feld der Atome des Materials abgelenkt und abgebremst. Das Spektrum der entstehenden Strahlung ist kontinuierlich mit der maximal Energie der $ \gamma $-Quanten gleich der gesamten kinetischen Energie des Elektrons.
\section{Wechselwirkung von elektromagnetischer Strahlung mit Materie}
Die Strahlung gibt hauptsächlich über die folgenden drei Effekte ihre Energie ab.
\subsection{Photoeffekt}
Beim Photoeffekt gibt ein Photon seine gesamte Energie an ein Elektron ab. Die Energie, die nach dem herauslösen aus dem Verband $ W_k $ übrig ist geht in Form kinetischer Energie an das Elektron über. Dieser Effekt dominiert bei kleinen Energien bis zu einigen hundert keV.
\subsection{Compton-Effekt ?????}
Beim Compton-Effekt stößt das Photon mit einem ruhenden Elektron elastisch. Die Energie des Photons und des Elektrons nach dem Stoß ist 
\begin{align}
E_\gamma'(\theta)&=\frac{E_\gamma}{1+\frac{E_\gamma}{m_ec^2}(1-\cos\theta)}\\
E_e'(\theta)&=E_\gamma-E_\gamma'(\theta)
\end{align}
Dieser Effekt dominiert zwischen einigen hundert und tausend keV. Aus dieser Formel ist allerdings nicht ersichtlich unter welchem Winkeln die Photonen bevorzugt gestreut werden. Dies wird durch den Klein-Nishina-Wirkungsquerschnitt beschrieben:
\begin{align}
\frac{\text{d}\sigma}{\text{d}\Omega}=\frac{r_0^2}{2}\left(\frac{k}{k_0}\right)^2\cdot\left(\Phi_0+f\cdot\text{P}\cdot\Phi_\text{H}\right)
\end{align}
Mit 
\begin{align*}
\Phi_0&=1+\cos^2\omega+(k_0-k)(1-\cos\omega)\\
\Phi_\text{H}&=-(1-\cos\omega)[(k_0+k)\cos\omega\cdot\cos\phi+k\cdot\sin\omega\cdot\sin\psi\cdot\cos\phi]\\
r_0&:\text{ klassischer Elektronenbahnradius}\\
k_0,k&:\text{ Photonenimpuls vor und nach dem Stoß in einheiten von }m_ec^2\\
f&:\text{ Anteil der ausgerichteten Eisenelektronen}\\
\text{P}&:\text{ Polarisationsgrad der Photonen und Elektronen}\\
\omega&:\text{ Streuwinkel (zwischen }k_0\text{ und }k)\\
\psi&:\text{ Winkel zwischen }k_0\text{und dem Spin der Elektronen der Eisenatome}\\
\phi&:\text{ Winkel zwischen }k_0-k\text{ Ebene }k_0-\text{Spin Ebene}
\end{align*}
Interessant für uns ist hier die Abhängigkeit vom Polarisationsgrad P, denn über den von uns gemessenen Zählrateneffekt $\eta=\text{P}f\frac{\Phi_0}{\Phi_\text{H}}$ können wir so auf die Polarisation der Elektronen schließen (siehe Abschnitt \ref{subsec:polgrad}).
\subsection{Paarbildung}
Ab einer Energie von 1022 keV kann Paarbildung auftreten. Dabei wird das Photon in ein Elektron und ein Positron umgewandelt und die überschüssige Energie wird dabei an die Teilchen weitergegeben. Dominant wird dieser Effekt ab einigen tausend keV.
\section{Szintillationsdetektor}
Auf die drei beschriebenen Weisen wird die Energie nun im Detektor deponiert. Geht die gesamte Energie in den Detektor über, trägt dies zum Photopeak bei. Verlässt ein compton-gestreutes Photon den Detektor so entkommt ein Teil der Energie und das Photon trägt zum Compton-Kontinuum bei. Das Positron aus der Paarbildung zerstrahlt wieder zu zwei Photonen. Werden beide (oder eines) nicht detektiert, so wird nur die Energie $ E_\gamma-1022 $keV ($ E_\gamma-511 $keV) gemessen, der sogenannte Double-Escape-Peak (Single-Escape-Peak).
\section{Wu-Experiment bene}
Einen Nachweis der Paritätsverletzung bei der schwachen Wechselwirkung liefert das Wu-Experiment. Dabei wird der Betazerfall an \isotope[60]{Co} untersucht, welches zu \isotope[60]{Ni}

\section{Zerfallschema bene}
	
\chapter{Durchführung und Aufbau bene}
Der Aufbau ist in Abbildung    dargestellt. 
  
     	
\chapter{Auswertung}
Zuerst wird das Eichspektrum des Natriums und anschließend das Bremsspektrum des Strontiums ausgewertet.
\section{Natrium}
Zuerst wird den Kanalnummern durch die Energieeichung eine Energie zugeordnet und anschließend überprüft, ob eine apparativ vorgetäuschte Polarisation vorliegt.
\subsection{Energiekalibrierung}
Das mit \isotope[22]{Na} aufgenommene Spektrum ist in Abbildung \ref{fig:kanalK} zu sehen.
\begin{figure}[H]
\centering
   	\begin{minipage}[b]{\textwidth}
   	\includegraphics[width=\textwidth]{Graphen/kanalzsmK.pdf}
   	\caption{Zählraten der zwei ADC, man erkennt, dass diese sich kaum unterscheiden. In rot die Kanalzusammenfassung von ADC1}
  	\label{fig:kanalK}
   	\end{minipage}
\end{figure}
Der Peak bei Kanal 591 ist der Photopeak vom verwendeten \isotope[22]{Na}, davor ist die Compton-Kante zu erkennen. Der Peak bei Kanal 213 entspricht der Ruheenergie des beim Zerfall entstanden Positrons von 511 keV. Die Peaks unterhalb von Kanal 200 sind Artefakte aus der Messelektronik. Das Untergrundspektrum ist hier bereits abgezogen.??\\
Mit diesen zwei Peaks kann nun eine Energiekalibrierung durchgeführt werden. Wir wählen $ f(K)=mK+b $. Es ergibt sich $ m=(2,021\pm0,015)\frac{\text{keV}}{\text{Kanal}} $ und $ b=(80,492\pm5,152) \text{keV}$. Damit ergibt sich das Spektrum in Abbildung \ref{fig:kanal}.
\begin{figure}[H]
\centering
   	\begin{minipage}[b]{\textwidth}
   	\includegraphics[width=\textwidth]{Graphen/kanalzsm.pdf}
   	\caption{Eichspektrum aufgetragen über die Energie}
  	\label{fig:kanal}
   	\end{minipage}
\end{figure}
\subsection{Apparativ vorgetäuschte Polarisation}
Um zu untersuchen ob apparativ vorgetäuschte Polarisation vorliegt berechnen wir den Zählrateneffekt $ \eta=\frac{z^+-z^-}{z^++z^-} $ dieser müsste symmetrisch um die Null verteilt sein, da keine Polarisation vorliegen sollte.
\begin{figure}[H]
\centering
   	\begin{minipage}[b]{\textwidth}
   	\includegraphics[width=\textwidth]{Graphen/eta.pdf}
   	\caption{Zählrateneffekt $ \eta $ für $ \isotope[22]{Na} $}
  	\label{fig:eta}
   	\end{minipage}
\end{figure}
Wir betrachten die Kanäle 21 bis 760, da außerhalb die Zählraten selbst mit Kanalzusammenfassung zu niedrig für eine statistisch sinnvolle Aussage sind. Auf den ersten Blick sehen die Werte symmetrisch verteilt aus. Um dies zu überprüfen führen wir einen $ \chi^2 $-Test durch. Die Prüfgröße $ \chi^2 $ ist
\begin{equation}
\chi^2=\sum\limits_i \frac{(\eta_i-x_i)^2}{\sigma^2}
\end{equation}
Hier ist $ x_i $ der erwartete Wert für den jeweiligen Wert und $ \sigma^2 $ die Varianz. In unserem Fall ist $ x_i\equiv 0 $ für alle $i$. Um das reduzierte $ \chi^2 $ zu erhalten müssen wir durch die Anzahl der Freiheitsgrade teilen, die bei uns der Anzahl der Kanäle minus 1 entspricht, da ein Freiheitsgrad verloren geht, wenn wir die Varianz berechnen. Es ergibt sich 
\begin{equation}
\chi^2_{red]}=\frac{\chi^2}{f-1}=\frac{743,26}{739-1}=1,00712
\end{equation}
Damit liegt keine apparativ vorgetäuschte Polarisation vor.
\section{Strontium}
\subsection{Bremsstrahlungsspektrum}
\subsection{Enpunktsenergie im Bremsstrahlungsspektrum}
\subsection{Polarisationsgrad}\label{subsec:polgrad}
\section{Fehlerberechnung und Messzeitabschätzung}

\chapter{Fazit}


\chapter{Anhang}





		

\renewcommand{\bibname}{Literaturverzeichnis}
\begin{thebibliography}{Bak89}



\end{thebibliography} 	



\end{document} 