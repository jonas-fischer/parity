\documentclass[twoside,colorback,accentcolor=tud4c,11pt]{tudreport}
\usepackage{tablefootnote}
\usepackage{ngerman}
\usepackage[utf8]{inputenc} 
\usepackage[T1]{fontenc}
\usepackage{graphicx}
\usepackage{isotope}
\usepackage{tabularx}
\usepackage{tabulary}
\usepackage{float}
\usepackage{siunitx}
\usepackage{hyperref}
\usepackage{siunitx}
\usepackage{units}
\usepackage{upgreek}
\usepackage{biblatex}
\usepackage[figure]{hypcap}
\usepackage{multirow}
\usepackage{amsmath}
\usepackage{subfig}
\usepackage{xfrac}
\usepackage{gensymb}
\usepackage{braket}

\title{Paritätsverletzung beim Beta-Zerfall 2.7}
\subtitle{	\begin{tabular}{p{8cm}ll}
Benedikt Paul Schallmo   &   Jonas Fischer \\ Matrikelnummer: 2686286  &   Matrikelnummer: 2240758       \\ email: \textaccent{ benediktschallmo@yahoo.de} & email: \textaccent{jonas.fischer.42gmail.com}  
			\end{tabular} }
\subsubtitle{ \\Versuchsbetreuung: Kristian König\\ Datum der Durchführung: 24.04.2017 \\ Abgabetermin: 16.05.2017}
\institution{Praktikum für Fortgeschrittene}
\sponsor{Hiermit erklären wir, dass wir die vorliegende Arbeit bzw. Leistung eigenständig, ohne fremde Hilfe und nur unter Verwendung der angegebenen Hilfsmittel angefertigt haben. Alle übernommenen Textstellen aus der Literatur beziehungsweise dem Internet sind als solche kenntlich gemacht. Diese Arbeit hat in gleicher oder ähnlicher Form noch keiner Prüfungsbehörde vorgelegen. \\\\ 
\begin{tabular}{lp{2em}lp{2em}l}
 \hspace{4cm}   && \hspace{4cm}  && \hspace{4cm}
 \\\cline{1-1}\cline{3-3}\cline{5-5}
 Ort, Datum     && Benedikt Schallmo && Jonas Fischer
\end{tabular}  }


\dedication{}
\lowertitleback{}
\listfiles
    
\begin{document}

\maketitle 

\tableofcontents


\chapter{Einleitung und Ziel des Versuchs}
Die Tatsache, dass die Parität, anders als zunächst angenommen, bei physikalische Prozessen, bei denen schwache Wechselwirkung beteiligt ist, nicht erhalten bleibt, wird als Paritätsverletzung bezeichnet. Die ersten Anzeichen, dass die Parität bei der schwachen Wechselwirkung nicht erhalten sein könnte, lieferte das $\Theta$-$\tau$-Rätsel, bei dem der Zerfall des K-Mesons untersucht wurde. Die Physikerin Wu konnte die Paritätsverletzung erstmals durch die Asymmetrie der $\beta$-Emission an ausgerichteten Kernen nachweisen. Während des Versuchs soll ein technisch weniger aufwändiger Aufbau verwendet werden. Das Spektrum von Natrium wird zunächst zur Kalibrierung aufgenommen. Anschließend wird der Untergrund aufgenommen. Für den eigentlichen Versuch wird Strontium, und dessen Zerfallsprodukt Yttrium, als Betastrahler verwendet. Der Nachweis der Paritätsverletzung erfolgt mit Hilfe eines Comptonpolarimeters, mit dem die Polarisation der Bremsstrahlung der Betastrahlung nachgewiesen werden kann. Dazu wird zunächst die Endpunktsenergie der Bremsstrahlung experimentell bestimmt und mit der Theorie verglichen. Im Anschluss soll über den Polarisationsgrad $P(v/c)$ der Grad der Paritätsverletzung bestimmt werden.
\chapter{Physikalische Grundlagen}
\section{Radioaktive Zerfälle}
\subsection{Alphazerfall}
Beim Alphazerfall handelt es sich um einen radioaktiven Zerfallsprozess, bei dem spontan ein Alphastrahler unter Emission eines Alpha-Teilchens (Helium-4-Kern, bestehend aus 2 Protonen und 2 Neutronen) umgewandelt wird. Für den Zerfallsprozess gilt:
\begin{align*}
\isotope[A][Z]{X}\longrightarrow\isotope[A-4][Z-2]{Y}+\isotope[4][2]{He} + \Delta E,
\end{align*}
wobei \isotope[A][Z]{X} das Mutternuklid, \isotope[A-4][Z-2]{Y} das Tochternuklid und $\Delta E$ die freiwerdende Energie (kinetische Energie des Alphateilchens) bezeichnet. Die Alphastrahlung besitzt dabei einen für den jeweiligen Übergang charakteristischen Energiebetrag.
\subsection{Betazerfall}
Beim Betazerfall handelt es sich um einen Zerfall aufgrund der schwachen Wechselwirkung, wobei man unterscheidet den Beta-Minus-Zerfall und den Beta-Plus-Zerfall.
Im Gegensatz zur Alphastrahlung besitzt die kinetische Energie der Betateilchen einen bis zu einem für den Zerfall charakteristischen Maximalwert ein kontinuierliches Spektrum, da die freiwerdende Energie auf drei Teilchen verteilt werden kann. 
Beim Beta-Plus-Zerfall wird ein Proton im Kern in ein Neutron umgewandelt:
\begin{align*}
\isotope[1][1]{p}\rightarrow \isotope[1][0]{n}+e^+ + \nu_e. 
\end{align*}
Im Quarkmodell handelt es sich hierbei um die Umwandlung eines u-Quark in ein d-Quark:
\begin{align*}
(\text{uud})\rightarrow (\text{udd})+e^+ + \nu_e. 
\end{align*}
Beim Beta-Minus-Zerfall wird ein Neutron im Kern in ein Proton umgewandelt:
\begin{align*}
\isotope[1][0]{n}\rightarrow \isotope[1][1]{p}+e^- + \bar{\nu}_e. 
\end{align*}
Im Quarkmodell handelt es sich hierbei um die Umwandlung eines d-Quark in ein u-Quark:
\begin{align*}
(\text{udd})\rightarrow (\text{uud})+e^+ + \nu_e. 
\end{align*}
Zu den Betazerfällen gehört auch der Elektroneneinfang, bei dem ein Elektron der Atomhülle vom Kern absorbiert wird und ein Proton in ein Neutron umgewandelt wird. Zudem wird ein Elektron-Neutrino emittiert.
\begin{align*}
\isotope[1][1]{p}+e^-\rightarrow \isotope[1][0]{n} + \nu_e.
\end{align*}
\subsection{Andere Zerfallsarten}
Zu anderen Zerfallsarten gehört der Gammazerfall, bei dem, unter Aussendung eines Gammaquant, ein Atom in einem angeregten Zustand in seinen Grundzustand übergeht. Des Weiteren gibt es noch andere Prozesse, wie zum Beispiel die Emission von einem oder mehreren Protonen beziehungsweise Neutronen oder den Clusterzerfall, bei dem ein kleinerer Kern (\isotope[14]{C}, \isotope[28]{Si}) emittiert wird. 
\section{Die vier Wechselwirkungen}
Es existieren vier fundamentale Wechselwirkungen: Die Gravitation, der Elektromagnetismus, die schwache Wechselwirkung und die starke Wechselwirkung. 
Die verschiedenen Wechselwirkungen unterscheiden sich im Kopplungsmechanismus und den Austauschteilchen, und somit in der Reichweite und Stärke. Für jede Wechselwirkung existieren Austauschteilchen mit ganzzahligem Spin. Das Photon dient als Austauschteilchen bei der elektromagnetischen Wechselwirkung, das (postulierte) Graviton bei der Gravitation, drei Bosonen ($W^+$,$W^-$ und $Z^0$) bei der schwachen Kernkraft und 8 Gluonen bei der starken Kernkraft. Sowohl das Graviton als auch das Photon besitzen keine Ruhemasse, sodass die Reichweiten der jeweiligen Wechselwirkungen nicht beschränkt sind. Die Gravitation koppelt an die Masse und die elektromagnetische Kraft an die Ladung. Die starke Wechselwirkung kann mit der Quantenchromodynamik beschrieben werden: Alle Hadronen bestehen aus zwei oder drei Quarks. Diese Quarks besitzen einen eigenen Typ Ladung (Farbladung), die in drei verschiedenen Farben auftritt. Diese Farbladung kommt nicht isoliert vor, das heißt Quarks und Gluonen können nicht als freie Teilchen auftreten, sodass, obwohl auch die Gluonen keine Masse besitzen, ist die starke Kernkraft auf den Radius eines Hadrons beschränkt ($\approx 10^{-15}$). Im Gegensatz zur starken Wechselwirkung, bei der die Farbladung
eines Quarks sich ändert, der Quarktyp aber erhalten
bleibt, ändert sich bei der schwachen Wechselwirkung der
Quarktyp. Die schwache Wechselwirkung beruht auf der schwachen Ladung und da die beteiligten Austauschteilchen eine relativ große Masse besitzen ist die Kraft nur kurzreichweitig. \\ Die Reichweiten und die Kräfte sind in Abbildung \ref{V_2.2} dargestellt.
\begin{figure}[H]
\centering
   	\begin{minipage}[b]{0.6\textwidth}
   	\includegraphics[width=1\textwidth]{Graphen/grudkrafte.pdf}
   	\caption{Vergleich der fundamentalen Kräfte [2]}
  	\label{V_2.2}
   	\end{minipage}
\end{figure}


\paragraph{Erhaltungsgrößen}
Für die Wechselwirkungen sind Erhaltungsgrößen von entscheidender Bedeutung. Bei allen Prozessen bleiben charakteristischen Eigenschaften wie Energie, Impuls, Drehimpuls, die Baryonenzahl B und die Leptonenzahl
L konstant. Die Parität P, die Ladungsquantenzahl C und die Zeitspiegelinvarianz T, sowie das Produkt C$\cdot$P, sind bei der schwachen Wechselwirkung nicht erhalten. Das Produkt C$\cdot$P$\cdot$T bleibt jedoch auch bei der schwachen Wechselwirkung erhalten.\\
Lee und Young postulierten daraufhin 1956, dass die Parität vom schwachen Zerfall verletzt wird, was Wu ein Jahr später experimentell bestätigen konnte.
\section{Bremsstrahlung}
Bremsstrahlung bezeichnet die elektromagnetische Strahlung, die entsteht, wenn elektrische Ladung beschleunigt wird. In unserem Fall handelt es sich um ein Abbremsen in Materie. Die Elektronen werden im Feld der Atome des Materials abgelenkt und abgebremst. Das Spektrum der entstehenden Strahlung ist kontinuierlich mit der maximal Energie der $ \gamma $-Quanten gleich der gesamten kinetischen Energie des Elektrons.
\section{Wechselwirkung von elektromagnetischer Strahlung mit Materie}
Die Strahlung gibt ihre Energie hauptsächlich über die folgenden drei Effekte ab.
\subsection{Photoeffekt}
Beim Photoeffekt gibt ein Photon seine gesamte Energie an ein Elektron ab. Die Energie, die nach dem herauslösen aus dem Verband übrig ist geht in Form kinetischer Energie an das Elektron über. Dieser Effekt dominiert bei kleinen Energien bis zu einigen hundert keV.
\subsection{Compton-Effekt}
Beim Compton-Effekt stößt das Photon mit einem ruhenden Elektron elastisch. Die Energie des Photons und des Elektrons nach dem Stoß ist 
\begin{align}
E_\gamma'(\theta)&=\frac{E_\gamma}{1+\frac{E_\gamma}{m_ec^2}(1-\cos\theta)}\\
E_e'(\theta)&=E_\gamma-E_\gamma'(\theta)\label{eq:comton}
\end{align}
Dieser Effekt dominiert zwischen einigen hundert und tausend keV. Aus dieser Formel ist allerdings nicht ersichtlich unter welchem Winkeln die Photonen bevorzugt gestreut werden. Dies wird durch den Klein-Nishina-Wirkungsquerschnitt beschrieben:
\begin{align}
\frac{\text{d}\sigma}{\text{d}\Omega}=\frac{r_0^2}{2}\left(\frac{k}{k_0}\right)^2\cdot\left(\Phi_0+f\cdot P\cdot\Phi_\text{H}\right)
\end{align}
Mit 
\begin{align}
\Phi_0&=1+\cos^2\omega+(k_0-k)(1-\cos\omega)\label{eq:phi0}\\
\Phi_\text{H}&=-(1-\cos\omega)[(k_0+k)\cos\omega\cdot\cos\phi+k\cdot\sin\omega\cdot\sin\psi\cdot\cos\phi]\label{eq:phih}
\end{align}
\begin{align*}
r_0&:\text{ klassischer Elektronenbahnradius}\\
k_0,k&:\text{ Photonenimpuls vor und nach dem Stoß in Einheiten von }m_ec^2\\
f&:\text{ Anteil der ausgerichteten Eisenelektronen}\\
P&:\text{ Polarisationsgrad der Photonen und Elektronen}\\
\omega&:\text{ Streuwinkel (zwischen }k_0\text{ und }k)\\
\psi&:\text{ Winkel zwischen }k_0\text{und dem Spin der Elektronen der Eisenatome}\\
\phi&:\text{ Winkel zwischen }k_0-k\text{ Ebene }k_0-\text{Spin Ebene}
\end{align*}
Interessant für uns ist hier die Abhängigkeit vom Polarisationsgrad P, denn über den von uns gemessenen Zählrateneffekt $\eta=Pf\frac{\Phi_0}{\Phi_\text{H}}$ können wir so auf die Polarisation der Elektronen schließen.
\subsection{Paarbildung}
Ab einer Energie von 1022 keV kann Paarbildung auftreten. Dabei wird das Photon in ein Elektron und ein Positron umgewandelt und die überschüssige Energie wird dabei an die Teilchen weitergegeben. Dominant wird dieser Effekt ab einigen tausend keV.
\section{Szintillationsdetektor}
Auf die drei beschriebenen Weisen wird die Energie nun im Detektor deponiert. Geht die gesamte Energie in den Detektor über, trägt dies zum Photopeak bei. Verlässt ein compton-gestreutes Photon den Detektor so entkommt ein Teil der Energie und das Photon trägt zum Compton-Kontinuum bei. Das Positron aus der Paarbildung zerstrahlt wieder zu zwei Photonen. Werden beide (oder eines) nicht detektiert, so wird nur die Energie $ E_\gamma-1022 $keV ($ E_\gamma-511 $keV) gemessen, der sogenannte Double-Escape-Peak (Single-Escape-Peak). Je größer die Primärenergie, desto mehr Elektronen werden im Detektor ausgelöst. Die Energieauflösung hängt einerseits vom verwendeten Detektormaterial ab, aber auch von der angeschlossenen Elektronik. So ist die Anzahl der Kanäle entscheidend für die Auflösung, wobei ein Kompromiss zwischen hoher Kanalanzahl oder hoher Zählrate pro Kanal gefunden werden muss.
\section{Parität}
Parität beschreibt das Verhalten einer physikalischen Größe unter Raumspiegelung. So gilt etwa für den Ortsvektor $ \vec{r} $
\begin{align}
\hat{P}\ket{\vec{r}}=-\ket{\vec{r}}
\end{align}
Dies nennt man negative Parität. Wenn sich das Vorzeichen nicht ändert besitzt diese Größe positive Parität, wie etwa der Drehimpuls $ \vec{L} $
\begin{align}
\vec{L}=m\vec{r}\times\vec{v}
\end{align}
Da sich sowohl bei $ \vec{r} $ als auch bei $ \vec{v} $ das Vorzeichen unter Raumspiegelung ändert ist 
\begin{align}
\hat{P}\ket{\vec{L}}=+\ket{\vec{L}}
\end{align}
\subsection{Polare und axiale Vektoren, Skalare und Pseudoskalare}
Als polaren Vektor bezeichnet man Vektoren, die unter Raumspiegelung ihr Vorzeichen ändern.\\
Als axiale Vektoren bezeichnet man folglich solche die dies nicht tun.\\
Ähnlich verhält es sich mit den Skalaren und Pseudoskalaren. Skalare ändern ihr Vorzeichen nicht, Pseudoskalare hingegen schon (siehe Tabelle \ref{tab:polar}).
\begin{table}[H]
\centering
\begin{tabular}{|c|c|c|}
\hline 
 & Parität + & Parität - \\ 
\hline 
Vektor & \shortstack{axial\\$\vec{\omega},\vec{L}$} & \shortstack{polar \\$\vec{v},\vec{x}$} \\ 
\hline 
Skalar & \shortstack{Skalar \\$m,T$} & \shortstack{Pseudoskalar\\Helizität $h$} \\ 
\hline 
\end{tabular} 
\caption{Beispiele für Polare und axiale Vektoren, Skalare und Pseudoskalare. Die Helizität $h$ ist definiert als $h=\vec{J}\cdot\frac{\vec{p}}{|\vec{p}|}$}\label{tab:polar}
\end{table}
Für uns kommen also nur polare Vektoren oder Pseudoskalare in Frage, wenn wir eine Paritätsverletzung messen wollen. Es bietet sich für uns an Pseudoskalare zu messen, da dies einfacher ist als Vektoren zu bestimmen.
\subsection{$\Theta$-$\tau$-Rätsel}
Lange nahm man an, dass die Parität eine Erhaltungsgröße sei, wie etwa die Energie oder der Impuls. Die Entdeckung des $ \Theta^+ $- und des $ \tau^+ $-Teilchen stellen dies in Frage. Die gleichen Eigenschaften der Teilchen legen nahe, dass es sich um ein und das selbe Teilchen handelt, jedoch zerfallen diese unterschiedlich:
\begin{align}
\Theta^+&\rightarrow\pi^++\pi^0\\
\tau^+&\rightarrow\pi^++\pi^++\pi^-
\end{align}
was an sich noch kein Problem ist, allerdings besitzen die Pionen alle eine negative Parität, womit sich für $\Theta^+$ eine positive und für $\tau^+$ eine negative Parität ergibt. Damit müssen die Teilchen als unterschiedliche behandelt werden oder die Parität ist beim Zerfall des \glqq $\Theta$-$\tau$-Teilchens\grqq\, nicht erhalten.
\section{Wu-Experiment}
Einen Nachweis der Paritätsverletzung bei der schwachen Wechselwirkung liefert das Wu-Experiment. Dabei wird der Betazerfall an \isotope[60]{Co} untersucht, welches zu \isotope[60]{Ni} zerfällt. Wichtig für den Nachweis ist dabei der genaue Zerfallsprozess, im Besonderen die Spinausrichtung der beteiligten Kerne von Bedeutung. Richtet man die Cobaltatome mit Hilfe eines Magnetfeldes in ihre Vorzungsrichtung aus, so zeigt der Spinvektor in Richtung des Magnetfeldes. Um diese Ausrichtung zu erreichen, müssen die Cobaltatome auf wenige Millikelvin herunter gekühlt werden. Das verwendete Isotop besitzt den Spin $S=5$ in Richtung des Magnetfeldes. Der Tochterkern besitzt einen Spin $S=4$ ebenfalls entlang des Magnetfeldes ausgerichtet. Somit sind die Spinrichtungen des Elektrons und des Antineutrinos durch die Drehimpulserhaltung eindeutig bestimmt. Im folgenden wird dann mit zwei unterschiedlichen Versuchsaufbauten gemessen. Zunächst 
wird die Zählrate der Elektronen entgegengesetzt der Richtung des Magnetfelds (und somit entgegengesetzt der Richtung des Cobaltspins) gemessen. Der Spin des Elektrons steht in diesem Fall entgegengesetzt seiner Ausbreitungsrichtung. Um die Paritätsverletzung nachzuweisen, muss der Aufbau danach gespiegelt werden. Um dies zu erreichen, wird das Magnetfeld umgekehrt. Da dieses Paritätsunabhängig ist entspricht dies dem Fall, dass der Aufbau gespiegelt wird und das Magnetfeld gleich ausgerichtet bleibt. Der Spin der gemessenen Elektronen steht in diesem in die gleiche Richtung wie dessen Ausbreitungsrichtung. Bleibt die Parität beim Betazerfall erhalten, müssen beide Fälle gleich wahrscheinlich sein, sodass die Zählraten übereinstimmen. Der Versuch zeigte jedoch, dass die Elektronen bevorzugt so emittiert werden, dass der Spin des Elektrons und dessen Ausbreitungsrichtung entgegengesetzt stehen.
\newpage
\section{Zerfallschemata}
Die Zerfallsschemata der im Versuch verwendeten radioaktiven Präparate sind in Abbildungen \ref{V_2.8a} (Natium) und \ref{V_2.8b}(Strontium) abgebildet. \isotope[22]{Natrium} zerfällt unter Betazerfall (etwa 90\%) und unter Elektroneneinfang (etwa 10\%) in einen angeregten Zustand von \isotope[22]{Neon}. Dieses zerfällt unter Aussendung eines Gammaquant ($E=1274,542$ keV) in den Grundzustand.
\begin{figure}[H]
\centering
   	\begin{minipage}[b]{0.7\textwidth}
   	\includegraphics[width=1\textwidth]{Graphen/natrium.pdf}
   	\caption{Zerfallsschema Natrium [3]}
  	\label{V_2.8a}
   	\end{minipage}
\end{figure}
\isotope[90]{Strontium} zerfällt unter Beta-Minus-Zerfall zu \isotope[90]{Yttrium}. Die maximale Energie, die dabei auf ein Elektron übertragen werden kann beträgt $E_{\beta}=546,2$ keV. \isotope[90]{Yttrium} zerfällt ebenfalls unter Beta-Minus-Zerfall zu \isotope[90]{Zr}. Die maximale Energie des Elektrons beträgt bei diesem Zerfall $E_{\beta}=2283,9$ keV.
\begin{figure}[H]
\centering
   	\begin{minipage}[b]{0.7\textwidth}
   	\includegraphics[width=1\textwidth]{Graphen/strontium.pdf}
   	\caption{Zerfallsschema Strontium [3]}
  	\label{V_2.8b}
   	\end{minipage}
\end{figure}
	
	
\chapter{Durchführung und Aufbau}
Der Aufbau ist in Abbildung \ref{Aa} dargestellt. Die Strontium Quelle emittiert Elektronen. Diese werden im nachfolgenden Bleiabsorber abgebremst und emittieren somit Bremsstrahlung. Diese Bremsstrahlung der Elektronen mit longitudinal ausgerichtetem Spin ist in gleichem Drehsinn polarisiert wie der Spin der Elektronen. Der Polarisationsgrad der Bremsstrahlung wird mit Hilfe eines Comptonpolarimeters gemessen. Dabei werden ausgerichtete Elektronen in magnetisiertem Eisen genutzt, da der Wirkungsquerschnitt für Compton-Streuung von der Spin- und der Polarisationsrichtung abhängt. Sind der Spin des Gammaquants und des Elektrons antiparallel, so kann das Gammaquant durch Umklappen des Spins des Elektrons leichter Energie abgeben, sodass die Streuung in diesem Fall stärker ist. Der Bleikonus in der Mitte dient zur Abschirmung der Primärstrahlung. Als Detektor dient ein NaJ-Kristall. Im Gegensatz zum Wu-Experiment muss die Probe nicht herabgekühlt werden. Die Paritätsverletzung wird direkt über die Helizität der Elektronen nachgewiesen. Dazu werden die Zählraten bei entgegengesetzten Einstellungen des Magnetfelds gemessen.


  
\begin{figure}[H]
\centering
   	\begin{minipage}[b]{1\textwidth}
   	\includegraphics[width=1\textwidth]{Graphen/aufbau.pdf}
   	\caption{Schematischer Versuchsaufbau [3]}
  	\label{Aa}
   	\end{minipage}
\end{figure}  


Unter der Annahme, dass die Parität erhalten bleibt, werden gleich viele Elektronen mit positiver (Spin zeigt in Ausbreitungsrichtung) und mit negativer (Spin zeigt entgegen der Ausbreitungsrichtung) Helizität emittiert. Misst man in diesem Fall die Zählraten bei entgegengesetzten Magnetfeldeinstellungen, so sollten diese gleich sein. Eine Asymmetrie in den Zählraten weist auf die Verletzung der Parität hin. 
     	     	
\chapter{Auswertung}
Erst wird das Kalibrierungsspektrum des Natriums und anschließend das Bremsspektrum des Strontiums bzw. Yttriums ausgewertet.
\section{Natrium}
Zunächst wird den Kanalnummern durch die Energieeichung eine Energie zugeordnet und anschließend überprüft, ob eine apparativ vorgetäuschte Polarisation vorliegt.
\subsection{Energiekalibrierung}
Das mit \isotope[22]{Na} aufgenommene Spektrum ist in Abbildung \ref{fig:kanalK} zu sehen.
\begin{figure}[H]
\centering
   	\begin{minipage}[b]{\textwidth}
   	\includegraphics[width=\textwidth]{Graphen/kanalzsmK.pdf}
   	\caption{Zählraten der zwei ADC, man erkennt, dass diese sich kaum unterscheiden. In rot die Kanalzusammenfassung von ADC1}
  	\label{fig:kanalK}
   	\end{minipage}
\end{figure}
Der Peak bei Kanal 591 ist der Photopeak vom verwendeten \isotope[22]{Na}, davor ist die Compton-Kante zu erkennen. Der Peak bei Kanal 213 entspricht der Ruheenergie des beim Zerfall entstanden Positrons von 511 keV. Die Peaks unterhalb von Kanal 200 sind Artefakte aus der Messelektronik. Das Untergrundspektrum ist hier bereits abgezogen.\\
Mit diesen zwei Peaks kann nun eine Energiekalibrierung durchgeführt werden. Wir wählen $ f(K)=mK+b $. Es ergibt sich $ m=(2,021\pm0,015)\frac{\text{keV}}{\text{Kanal}} $ und $ b=(80,492\pm5,152) \text{keV}$. Damit ergibt sich das Spektrum in Abbildung \ref{fig:kanal}.
\begin{figure}[H]
\centering
   	\begin{minipage}[b]{\textwidth}
   	\centering
   	\includegraphics[width=\textwidth]{Graphen/kanalzsm.pdf}
   	\caption{Eichspektrum aufgetragen über die Energie}
  	\label{fig:kanal}
   	\end{minipage}
\end{figure}
\subsection{Apparativ vorgetäuschte Polarisation}
Um zu untersuchen ob apparativ vorgetäuschte Polarisation vorliegt berechnen wir den Zählrateneffekt $ \eta=\frac{N_1-N_2}{N_1+N_2} $. Dieser müsste symmetrisch um die Null verteilt sein, da keine Polarisation vorliegen sollte.
\begin{figure}[H]
\centering
   	\begin{minipage}[b]{\textwidth}
   	\includegraphics[width=\textwidth]{Graphen/eta.pdf}
   	\caption{Zählrateneffekt $ \eta $ für $ \isotope[22]{Na} $}
  	\label{fig:eta}
   	\end{minipage}
\end{figure}
Wir betrachten die Kanäle 21 bis 760, da außerhalb die Zählraten selbst mit Kanalzusammenfassung zu niedrig für eine statistisch sinnvolle Aussage sind. Auf den ersten Blick sehen die Werte symmetrisch verteilt aus. Um dies zu überprüfen führen wir einen $ \chi^2 $-Test durch. Die Prüfgröße $ \chi^2 $ ist
\begin{equation}
\chi^2=\sum\limits_i \frac{(\eta_i-x_i)^2}{\sigma^2}
\end{equation}
Hier ist $ x_i $ der erwartete Wert für den jeweiligen Wert und $ \sigma^2 $ die Varianz. In unserem Fall ist $ x_i\equiv 0 $ für alle $i$. Um das reduzierte $ \chi^2 $ zu erhalten müssen wir durch die Anzahl der Freiheitsgrade teilen, die bei uns der Anzahl der Kanäle minus 1 entspricht, da ein Freiheitsgrad verloren geht, wenn wir die Varianz berechnen. Es ergibt sich 
\begin{equation}
\chi^2_{red]}=\frac{\chi^2}{f-1}=\frac{743,26}{739-1}=1,00712
\end{equation}
Damit liegt keine apparativ vorgetäuschte Polarisation vor.
\section{Strontium}
Mit dem Spektrum von Strontium soll die Paritätsverletzung nachgewiesen werden. Dazu muss zunächst die Enpunktsenergie bestimmt werden und im Anschluss über den Polarisationsgrad der Grad der Verletzung bestimmt werden.
\subsection{Bremsstrahlungsspektrum und Endpunktsenergie}
Abbildungen \ref{4.2.1a} und \ref{4.2.1b} zeigen die Bremsstrahlspektren aufgenommen am ADC1 und am ADC2. Zudem wurden je 10 Kanäle zusammengefasst, um die Statistik zu verbessern.  
\begin{figure}[H]
\centering
   	\begin{minipage}[b]{1\textwidth}
   	\includegraphics[width=1\textwidth]{Graphen/strontium1.pdf}
   	\caption{Bremsstrahlungsspektrum von Strontium am ADC1}
  	\label{4.2.1a}
   	\end{minipage}
\end{figure} 
Man kann erkennen, dass beide Bremsstrahlspektren sehr ähnlich sind. Durch bloßes betrachten lässt sich kein Unterschied feststellen. 
\begin{figure}[H]
\centering
   	\begin{minipage}[b]{1\textwidth}
   	\includegraphics[width=1\textwidth]{Graphen/strontium2.pdf}
   	\caption{Bremsstrahlungsspektrum von Strontium am ADC2}
  	\label{4.2.1b}
   	\end{minipage}
\end{figure}  

Die Endpunktsenergie bezeichnet die maximale Energie, die ein Bremsstrahlungsquant besitzen kann. Sie berechnet sich über die Comptonformel
\begin{align*}
E_{\text{End}}=\frac{E_{\beta}}{1+\dfrac{E_{\beta}}{m_ec^2}\left(1-\cos{\theta}\right)},
\end{align*}
dabei bezeichnet $E_{\beta}$ die Energie des Betazerfalls und $\theta=45,33$° den Streuwinkel. Für die Energie beim Zerfall von \isotope[90]{Strontium} gilt $E_{\beta}=546,2\,\si{keV}$, sodass sich für die Endpunktsenergie $E_{\text{End}}=414,6\,\si{keV}$ ergibt. Diese Energie ist im Bremsstrahlungsspektrum (siehe Abbildungen \ref{4.2.1a} und \ref{4.2.1b}) nicht zu erkennen. In Abbildungen \ref{4.2.1c} und \ref{4.2.1d} sind die Bremsstrahlungsspektren im Bereich von 700 keV bis 1100 keV vergrößert dargestellt. Da der Untergrund nur über einen sehr kurzen Zeitraum gemessen wurde, schwankt dieser sehr stark und es lässt sich nicht genau erkennen, ab wann das Spektrum in etwa dem Untergrund entspricht. Man kann lediglich erahnen, dass die experimentelle Endpunktsenergie bei etwa $E_{\text{End,exp.}}=850\,\si{keV} - 900\,\si{keV}$ liegt. Da der Untergrund nur über eine Zeitspanne von etwa 30 Minuten gemessen wurde und das Spektrum von Strontium über mehr als 27 Stunden gemessen wurde, können keine genaueren Angaben gemacht werden. Betrachtet man den Zerfall des \isotope[90]{Yttrium}, so erhält man für $E_{\beta}=2283,9\,\si{keV}$ für die Endpunktsenergie  $E_{\text{End}}=981,3\,\si{keV}$. Dies deckt sich im Rahmen der Genauigkeit mit den experimentellen Beobachtungen. 
\begin{figure}[H]
\centering
   	\begin{minipage}[b]{1\textwidth}
   	\includegraphics[width=1\textwidth]{Graphen/endenergie1.pdf}
   	\caption{Bremsstrahlungsspektrum von Strontium im Bereich von 700 bis 1100 keV am ADC1}
  	\label{4.2.1c}
   	\end{minipage}
\end{figure}  
\begin{figure}[H]
\centering
   	\begin{minipage}[b]{1\textwidth}
   	\includegraphics[width=1\textwidth]{Graphen/endenergie2.pdf}
   	\caption{Bremsstrahlungsspektrum von Strontium im Bereich von 700 bis 1100 keV am ADC2}
  	\label{4.2.1d}
   	\end{minipage}
\end{figure}
Für die Bestimmung des Polarisationsgrads werden im folgenden nur Energien unter etwa 875 keV verwendet.
\subsection{Polarisationsgrad}\label{subsec:polgrad}
Nun berechnen wir den Polarisationsgrad der $ \beta $-Strahlung. Dieser ist 
\begin{align}
P=\frac{\eta}{f\frac{\Phi_0}{\Phi_\text{H}}}
\end{align}
wobei wir $ \Phi_0$ und $\Phi_\text{H} $ mit \ref{eq:phi0} und \ref{eq:phih} berechnen und $ f $ gegeben ist durch
\begin{align}
f=\frac{\text{B}/\mu_0-\text{H}}{\text{Z}\cdot\text{N}_V\cdot\mu_{\text{B}}}=0,0478\pm 0,002
\end{align}
Die Zahlenwerte sind der Versuchsanleitung entnommen, $ \text{N}_V=\frac{\rho}{A_uu} $ mit der Atommasse von Eisen \\ $ A_u=55,845 $ u. 
Der Zählrateneffekt $ \eta $ ist 
\begin{align}
\eta=\frac{Z_1-Z_2}{Z_1+Z_2}
\end{align}
Hier sind die $ Z_i $ die Zählraten der verschiedenen Magnetfeldorientierungen. Der Polarisationsgrad der $ \beta $-Strahlung ist in Abbildung \ref{fig:P} über $ \beta=v/c $ aufgetragen. Die Geschwindigkeit der Elektronen erhalten wir aus deren Energie und diese wiederum \glqq rückwärts\grqq\, aus der Compton-Formel mit der gemessenen Photonenenergie (\ref{eq:comton}). Alle verwendeten Werte sind in Tabelle \ref{tab:P} aufgelistet.
\begin{figure}[H]
\centering
   	\begin{minipage}[b]{1\textwidth}
   	\includegraphics[width=1\textwidth]{Graphen/Pol.pdf}
   	\caption{Polarisation der $ \beta $-Strahlung bis zur Endpunktsenergie}
  	\label{fig:P}
   	\end{minipage}
\end{figure}
Die Theorie sagt eine Polarisation von 
\begin{align}
|P|=\beta
\end{align}
vorher. Dies entspricht einer Steigung von $ \pm 1 $ in Abbildung \ref{fig:P}. Bei uns ist die Steigung negativ aufgrund der Wahl von $ N_1 $ und $ N_2 $. Berücksichtigt man nur Werte unterhalb der abgelesenen Endpunktsenergie von $ E_\gamma\approx $ 875 keV, so erhält man eine Steigung von 
\begin{align}
m=-0,994\pm 0,109
\end{align}
Damit liegt der erwartete Wert innerhalb der $ 1\sigma $-Umgebung unseres Wertes.
\begin{table}[H]
\centering
\begin{tabular}{|c|c|c|c|c|c|c|c|c|c|}
\hline 
$ E_{\gamma} $/keV & $ E_{\beta} $/keV & $ N_1/10^3 $ &$ N_2/10^3 $&$ \beta $ & $ \eta $ & $ \Delta\eta $ & $ \Phi_{\text{H}}/\Phi_{\text{0}} $ & $ P $ & $ \Delta P $ \\
\hline
151.23 & 165.81 & 7737.50 & 7760.93 & 0.6557 & -0.0015 & 0.0003 & 0.0613 & -0.5163 & 0.0894 \\
\hline
171.44 & 190.42 & 7494.92 & 7542.57 & 0.6850 & -0.0032 & 0.0003 & 0.0699 & -0.9478 & 0.0867 \\
\hline
191.66 & 215.68 & 6622.17 & 6668.34 & 0.7110 & -0.0035 & 0.0003 & 0.0787 & -0.9230 & 0.0825 \\
\hline
211.87 & 241.62 & 5530.38 & 5566.79 & 0.7342 & -0.0033 & 0.0003 & 0.0876 & -0.7833 & 0.0788 \\
\hline
232.08 & 268.26 & 4446.26 & 4482.91 & 0.7550 & -0.0041 & 0.0003 & 0.0966 & -0.8885 & 0.0814 \\
\hline
252.29 & 295.64 & 3483.89 & 3519.55 & 0.7737 & -0.0051 & 0.0004 & 0.1057 & -1.0072 & 0.0858 \\
\hline
272.50 & 323.78 & 2691.61 & 2718.42 & 0.7908 & -0.0050 & 0.0004 & 0.1150 & -0.9018 & 0.0869 \\
\hline
292.71 & 352.72 & 2067.28 & 2095.40 & 0.8062 & -0.0068 & 0.0005 & 0.1243 & -1.1371 & 0.0952 \\
\hline
312.93 & 382.49 & 1589.29 & 1609.62 & 0.8203 & -0.0064 & 0.0006 & 0.1337 & -0.9945 & 0.0969 \\
\hline
333.14 & 413.12 & 1217.29 & 1236.88 & 0.8332 & -0.0080 & 0.0006 & 0.1432 & -1.1664 & 0.1053 \\
\hline
353.35 & 444.66 & 934.89 & 949.20 & 0.8450 & -0.0076 & 0.0007 & 0.1528 & -1.0399 & 0.1088 \\
\hline
373.56 & 477.15 & 716.05 & 729.40 & 0.8559 & -0.0092 & 0.0008 & 0.1625 & -1.1893 & 0.1181 \\
\hline
393.77 & 510.63 & 550.75 & 559.75 & 0.8659 & -0.0081 & 0.0009 & 0.1722 & -0.9839 & 0.1224 \\
\hline
413.98 & 545.14 & 422.74 & 429.76 & 0.8752 & -0.0082 & 0.0011 & 0.1821 & -0.9458 & 0.1306 \\
\hline
434.20 & 580.74 & 324.04 & 331.59 & 0.8837 & -0.0115 & 0.0012 & 0.1920 & -1.2543 & 0.1444 \\
\hline
454.41 & 617.48 & 249.20 & 254.81 & 0.8916 & -0.0111 & 0.0014 & 0.2020 & -1.1531 & 0.1536 \\
\hline
474.62 & 655.40 & 191.84 & 196.65 & 0.8989 & -0.0124 & 0.0016 & 0.2121 & -1.2214 & 0.1663 \\
\hline
494.83 & 694.58 & 146.93 & 150.51 & 0.9057 & -0.0120 & 0.0018 & 0.2222 & -1.1317 & 0.1790 \\
\hline
515.04 & 735.07 & 111.69 & 115.95 & 0.9120 & -0.0187 & 0.0021 & 0.2324 & -1.6817 & 0.2014 \\
\hline
535.25 & 776.94 & 86.96 & 88.91 & 0.9179 & -0.0111 & 0.0024 & 0.2426 & -0.9557 & 0.2095 \\
\hline
555.47 & 820.26 & 66.28 & 68.63 & 0.9234 & -0.0174 & 0.0027 & 0.2529 & -1.4424 & 0.2332 \\
\hline
575.68 & 865.12 & 51.22 & 52.89 & 0.9285 & -0.0161 & 0.0031 & 0.2632 & -1.2767 & 0.2521 \\
\hline
595.89 & 911.58 & 39.45 & 40.51 & 0.9333 & -0.0132 & 0.0035 & 0.2735 & -1.0083 & 0.2738 \\
\hline
616.10 & 959.75 & 30.43 & 31.21 & 0.9377 & -0.0127 & 0.0040 & 0.2838 & -0.9325 & 0.2994 \\
\hline
636.31 & 1009.71 & 23.30 & 24.00 & 0.9419 & -0.0147 & 0.0046 & 0.2942 & -1.0433 & 0.3298 \\
\hline
656.52 & 1061.57 & 18.14 & 18.95 & 0.9457 & -0.0219 & 0.0052 & 0.3046 & -1.5018 & 0.3621 \\
\hline
676.74 & 1115.43 & 14.17 & 14.44 & 0.9494 & -0.0094 & 0.0059 & 0.3149 & -0.6267 & 0.3935 \\
\hline
696.95 & 1171.43 & 10.91 & 11.58 & 0.9528 & -0.0299 & 0.0067 & 0.3253 & -1.9247 & 0.4362 \\
\hline
717.16 & 1229.68 & 8.87 & 8.97 & 0.9559 & -0.0053 & 0.0075 & 0.3357 & -0.3284 & 0.4669 \\
\hline
737.37 & 1290.32 & 7.23 & 7.20 & 0.9589 & 0.0019 & 0.0083 & 0.3460 & 0.1132 & 0.5035 \\
\hline
757.58 & 1353.51 & 5.75 & 5.88 & 0.9617 & -0.0120 & 0.0093 & 0.3562 & -0.7069 & 0.5453 \\
\hline
777.79 & 1419.41 & 4.83 & 4.76 & 0.9643 & 0.0073 & 0.0102 & 0.3665 & 0.4165 & 0.5830 \\
\hline
798.01 & 1488.20 & 3.87 & 3.99 & 0.9668 & -0.0149 & 0.0113 & 0.3767 & -0.8266 & 0.6273 \\
\hline
818.22 & 1560.06 & 3.45 & 3.32 & 0.9691 & 0.0191 & 0.0122 & 0.3868 & 1.0311 & 0.6588 \\
\hline
838.43 & 1635.22 & 2.86 & 3.03 & 0.9712 & -0.0299 & 0.0130 & 0.3968 & -1.5758 & 0.6899 \\
\hline
858.64 & 1713.91 & 2.52 & 2.51 & 0.9733 & 0.0020 & 0.0141 & 0.4068 & 0.1024 & 0.7256 \\
\hline
878.85 & 1796.37 & 2.25 & 2.31 & 0.9752 & -0.0136 & 0.0148 & 0.4167 & -0.6835 & 0.7445 \\
\hline
\end{tabular} 
\caption{Polarisation und Zählrateneffekt bis zur abgelesenen Endpunktsenergie}\label{tab:P}
\end{table}
\section{Messzeitabschätzung}
Der Fehler des Polarisationsgrads soll kleiner sein als 1\%. Der Einfachheit halber nehmen wir an, dass 
\begin{align}
\frac{\Delta P}{P}=\frac{\Delta \eta}{\eta}\\
\Rightarrow \frac{\Delta \eta}{\eta}\stackrel{!}{\leq} 0,01
\end{align}
Unter Berücksichtigung des Untergrunds ist 
\begin{align}
\eta&=\frac{N_1-N_2}{N_1+N_2-2U}\\
\Delta\eta&=\frac{2\sqrt{((N_2-U)\Delta N_1)^2+((N_1-U)\Delta N_2)^2+((N_1-N_2)\Delta U)^2}}{(N_1+N_2-2U)^2}
\end{align}
Es gilt $ N_i=Z_i\cdot t $ und $ (\Delta N_i)^2=N_i $, ebenso für $U$.\\
Mit den Werten von $ E_{\gamma,\text{end}}=879 $ keV ergeben sich
\begin{align}
Z_1=1,378 \text{ min}^{-1}\\
Z_2=1,416 \text{ min}^{-1}\\
\dot{U}=1,187\text{ min}^{-1}
\end{align}
und damit 
\begin{align}
t\geq 37,5 \text{ a}
\end{align}
Setzten wir umgekehrt den gemessenen relativen Fehler von $ \frac{\Delta \eta}{\eta}=1,088 $ ein, so erhalten wir die tatsächliche Messzeit von 1630 Minuten. Unser Ergebnis ist also mit Vorsicht zu genießen, da die Werte bei höheren Energien mit enormen Unsicherheiten behaftet sind.
\chapter{Fazit}
Zunächst konnte das Spektrum von Natrium erfolgreich aufgenommen werden und damit eine Energiekalibrierung durchgeführt werden. Dabei wurden zwei Peaks, einer bei 511 keV und der andere bei 1275 keV, verwendet. Im Anschluss wurde zur weiteren Auswertung der Untergrund aufgenommen. Für den eigentlichen Versuch wurden Strontium und Yttrium als Betastrahler verwendet. Der Nachweis der Paritätsverletzung erfolgte mit Hilfe des Comptonpolarimeters, mit dem die Polarisation der Bremsstrahlung der Betastrahlung nachgewiesen werden kann. Um eine durch die Apparatur vorgetäuschte Polarisation auszuschließen beziehungsweise zu eliminieren, wurde zunächst der Zählrateneffekt in Abhängigkeit von der Energie für Natrium bestimmt. Dabei ergaben sich keine Auffälligkeiten, sodass davon ausgegangen werden kann, dass die Apparatur keine Polarisation vortäuscht. Um den Zählrateneffekt von Strontium und Yttrium zu bestimmen, wurde zunächst die Endpunktsenergie der Bremsstrahlung experimentell bestimmt und mit der Theorie verglichen. Die bestimmte Energie lässt sich im Rahmen der Genauigkeit dem Zerfall von Yttrium zuordnen. Für die Bestimmung des Polarisationsgrades wurde somit als Endpunktsenergie etwa 875 keV angenommen. Die Auftragung des Polarisationsgrad $P(v/c)$ über $\beta$ zeigt, dass die Parität verletzt ist. Auch die These, dass die Polarisation $P$ proportional zu $\beta$ ist, liegt im Rahmen der Genauigkeit. Somit konnte die Paritätsverletzung im Rahmen des Versuches nachgewiesen werden. Abschließend sollte eine Messzeitabschätzung durchgeführt werden. Diese zeigt, dass die Ergebnisse nahe der Enpunktsenergie mit Vorsicht betrachtet werden müssen, da, um den relativen Fehler der Polarisation auf unter 1\% bestimmen zu können, über einen wesentlich längeren Zeitraum gemessen werden muss.



		

\renewcommand{\bibname}{Literaturverzeichnis}
\begin{thebibliography}{Bak89}
\bibitem[1]{1} Literaturmappe zum Versuch Paritätsverletzung beim Betazerfall, TU Darmstadt.
\bibitem[2]{2} Wolfgang Demtröder: Experimentalphysik 4: Kern-, Teilchen- und Astrophysik. 5.Auflage, Springer, 2017.
\bibitem[3]{3} Versuchsanleitung zum Versuch Paritätsverletzung beim Betazerfall, TU Darmstadt.
\bibitem[4]{4} Wolfgang Demtröder: Experimentalphysik 3: Atome, Moleküle und Festkörper. 4.Auflage, Springer, 2010.




\end{thebibliography} 	



\end{document} 
